% arara: lualatex: { shell: true }
% arara: lualatex: { shell: true }
% arara: lualatex: { shell: true, synctex: true }
% arara: convert: { options: ['-density', '600', '-colorspace', 'sRGB', 'main.pdf', 'main.png'] }
% arara: convert: { options: ['-density', '600', '-colorspace', 'sRGB', '../experiments/figures/friedman/statsplot_zoom.pdf', '-resize', '900x600', '-gravity', 'center', '-background', 'white', '-extent', '900x600', 'thumbnail.png' ] }
\RequirePackage{luatex85}
\documentclass[10pt]{article}

% Layout
\usepackage[a3paper,landscape]{geometry}

% Math support
\usepackage{mathtools,amssymb,dsfont}

% Fonts
\usepackage[factor=0]{microtype} % no protrusion
\usepackage[no-math]{fontspec}
\defaultfontfeatures{Ligatures=TeX}
% open source alternative to Gill Sans in UU's official layout
\usepackage[condensed,sfdefault]{gillius2}
\newfontfamily\JuliaMono{JuliaMono-Regular.otf}[
Path=../shared/fonts/,
Extension=.ttf
]
\newfontface\JuliaMonoMedium{JuliaMono-Regular}
\setmonofont{JuliaMonoMedium}[Contextuals=Alternate, Scale=MatchLowercase]
\usepackage[OT1,euler-digits]{eulervm}

% Languages
\usepackage{polyglossia}
\setdefaultlanguage{english}
\usepackage{csquotes}

% Graphics
\usepackage{graphicx}
\usepackage[export]{adjustbox}

% Tables
\usepackage{booktabs}

% Colors
\def\UseCMYK{true}
\ifdefined\UseCMYK
	\definecolor{uured}{cmyk}{0.07,1.00,0.82,0.26}
\else
	\definecolor{uured}{RGB}{166,25,46}
\fi

\ifdefined\UseCMYK
	\definecolor{black}{cmyk}{0.00,0.00,0.00,1.00}
\else
	\definecolor{black}{RGB}{0,0,0}
\fi

\colorlet{uulightgrey}{black!10}
\colorlet{uumidgrey}{black!30}
\colorlet{uudarkgrey}{black!50}

% Pantone 483
\ifdefined\UseCMYK
	\definecolor{mull}{cmyk}{0.21,0.80,0.81,0.69}
\else
	\definecolor{mull}{RGB}{101,48,36}
\fi
\colorlet{mullstark}{mull!30}
\colorlet{mullmellan}{mull!20}
\colorlet{mullsvag}{mull!10}

% Pantone 1675
\ifdefined\UseCMYK
	\definecolor{sand}{cmyk}{0.05,0.83,0.100,0.27}
\else
	\definecolor{sand}{RGB}{169,67,30}
\fi
\colorlet{sandstark}{sand!30}
\colorlet{sandmellan}{sand!20}
\colorlet{sandsvag}{sand!10}


% Pantone 131

\ifdefined\UseCMYK
	\definecolor{blond}{cmyk}{0.02,0.39,1.00,0.10}
\else
	\definecolor{blond}{RGB}{204,138,0}
\fi
\colorlet{blondstark}{blond!30}
\colorlet{blondmellan}{blond!20}
\colorlet{blondsvag}{blond!10}



% Pantone 7546
\ifdefined\UseCMYK
	\definecolor{skugga}{cmyk}{0.73,0.45,0.24,0.66}
\else
	\definecolor{skugga}{RGB}{37,55,70}
\fi
\colorlet{skuggastark}{skugga!30}
\colorlet{skuggamellan}{skugga!20}
\colorlet{skuggasvag}{skugga!10}

% Pantone 534
\ifdefined\UseCMYK
	\definecolor{skymning}{cmyk}{0.95,0.75,0.07,0.44}
\else
	\definecolor{skymning}{RGB}{37,54,93}
\fi
\colorlet{skymningstark}{skymning!30}
\colorlet{skymningmellan}{skymning!20}
\colorlet{skymningsvag}{skymning!10}

% Pantone 645
\ifdefined\UseCMYK
	\definecolor{gryning}{cmyk}{0.56,0.21,0.02,0.08}
\else
	\definecolor{gryning}{RGB}{125,161,196}
\fi
\colorlet{gryningstark}{gryning!30}
\colorlet{gryningmellan}{gryning!20}
\colorlet{gryningsvag}{gryning!10}

% Pantone 7484
\ifdefined\UseCMYK
	\definecolor{gronska}{cmyk}{0.91,0.14,0.78,0.60}
\else
	\definecolor{gronska}{RGB}{0,87,63}
\fi
\colorlet{gronskastark}{gronska!30}
\colorlet{gronskamellan}{gronska!20}
\colorlet{gronskasvag}{gronska!10}

% Pantone 7494
\ifdefined\UseCMYK
	\definecolor{skir}{cmyk}{0.35,0.05,0.42,0.14}
\else
	\definecolor{skir}{RGB}{156,175,136}
\fi
\colorlet{skirstark}{skir!30}
\colorlet{skirmellan}{skir!20}
\colorlet{skirsvag}{skir!10}

% QR code
\usepackage[nolinks]{qrcode}

% Better lists
\usepackage{enumitem}
\setlist{leftmargin=*,itemsep=0pt}
\setlist[itemize,1]{label={\color{uured}$\blacktriangleright$}}

% Boxes
\usepackage[poster,xparse,raster]{tcolorbox}
\tcbset{shield externalize}


% poster settings
\tcbposterset{
  coverage =
  {
    spread,
    interior style={color=white},
  },
  poster =
  {
    columns=10,
    rows=1,
  },
  boxes =
  {
    enhanced standard jigsaw,
    sharp corners,
    boxrule=0pt,
    leftrule=1pt,
    rightrule=1pt,
    lower separated=false,
    % Colors
    coltext=black,
    colback=white,
    colframe=black,
    coltitle=black,
    colbacktitle=uulightgrey,
    % Fonts
    fonttitle=\bfseries\large,
    % Subtitles
    subtitle style=
    {
      frame empty,
      hbox,
      coltext=white!50!uulightgrey,
      colback=black!10!uudarkgrey,
    },
  }
}

% Tikz settings
\usetikzlibrary{graphs,graphs.standard,graphdrawing,positioning,calc,patterns}
\usegdlibrary{force}
\usetikzlibrary{arrows}
\tikzset{line/.style={->,>=latex'}}

% Load PGFPlots
\usepackage{pgfplots}
\pgfplotsset{compat=1.17}

% Tufte style
\makeatletter
  \def\pgfplotsdataxmin{\pgfplots@data@xmin}
  \def\pgfplotsdataxmax{\pgfplots@data@xmax}
  \def\pgfplotsdataymin{\pgfplots@data@ymin}
  \def\pgfplotsdataymax{\pgfplots@data@ymax}
\makeatother
\pgfplotsset{
  range frame/.style={
    every axis legend/.append style={draw=none, fill=none, legend cell align=left},
    every axis/.append style={thick},
    tick style={thick,black},
    tick align=outside,
    scaled ticks=false,
    enlargelimits=false,
    axis lines*=left,
    line cap=round,
    clip=false,
    axis line shift=5pt,
    colorbar style={
      tick align=outside,
      ytick pos=right,
    },
  }
}

% Colorbrewer
\usepgfplotslibrary{colorbrewer}
\pgfplotsset{
  % Initialize Dark2-8:
  cycle list/Dark2-8,
  % Combine it with ’mark list*’:
  cycle multiindex* list={
    mark list*\nextlist
    Dark2-8\nextlist
    linestyles\nextlist
  },
}

% Units
\usepgfplotslibrary{units}
\pgfplotsset{unit code/.code 2 args={\text{#1#2}}}

\usepgfplotslibrary{groupplots,fillbetween}
\usetikzlibrary{positioning,calc,intersections}

% Define inverse hyperbolic sine
\pgfkeys{/pgf/declare function={arcsinh(\x) = ln(\x + sqrt(\x^2+1));}}

% plot statistic
\makeatletter
\newcommand{\@plotstats}[2][]{%
  \foreach \colormodel/\colorensemble/\data in {Dark2-A/Dark2-B/train, Dark2-C/Dark2-D/test}{%
    % path of minimum
    \edef\temp{%
      \noexpand\addplot [
        draw=none, name path=A, forget plot, prefix=pgfshell_, id={minimum_#2_\data},
      ] table shell {
        awk '
          BEGIN { FS = ","; };
          {
            if($2 == "#1" && $1 >= 10 && $1 <= 1500) {
              if(min[$1] == "" || $3 < min[$1]) min[$1] = $3;
            };
          };
          END { for(iteration in min) print iteration, min[iteration]; };
        ' ../experiments/data/friedman/statistics_id=*_dataset=\data.csv | sort -t, -k1 -n
      };

      % path of maximum
      \noexpand\addplot [
        draw=none, name path=B, forget plot, prefix=pgfshell_, id={maximum_#2_\data}
      ] table shell {
        awk '
          BEGIN { FS = ","; };
          {
            if($2 == "#1" && $1 >= 10 && $1 <= 1500) {
              if(max[$1] == "" || $3 > max[$1]) max[$1] = $3;
            };
          };
          END { for(iteration in max) print iteration, max[iteration]; };
        ' ../experiments/data/friedman/statistics_id=*_dataset=\data.csv | sort -t, -k1 -n
      };

      % fill between minimum and maximum
      \noexpand\addplot [color=\colormodel, opacity=0.2] fill between [of=A and B];

      % Mean
      \noexpand\addplot [
        color=\colormodel, prefix=pgfshell_, id={mean_#2_\data}
      ] table shell {
        awk '
          BEGIN { FS = ","; };
          {
            if($2 == "#1" && $1 >= 10 && $1 <= 1500) {
              sum[$1] += $3;
              counts[$1]++;
            };
          };
          END { for(iteration in sum) print iteration, sum[iteration]/counts[iteration]; };
        ' ../experiments/data/friedman/statistics_id=*_dataset=\data.csv | sort -t, -k1 -n
      };

      % ensembles
      \noexpand\addplot [
        color=\colorensemble, prefix=pgfshell_, id={ensemble_#2_\data}
      ] table shell {
        awk '
          BEGIN { FS = ","; };
          {
            if($2 == "#1" && $1 >= 10 && $1 <= 1500) print $1, $3;
          };
        ' ../experiments/data/friedman/statistics_ensembles_dataset=\data.csv | sort -t, -k1 -n
      };%
    }%
    \temp%
  }%
}
\newcommand\plotstats{\@dblarg\@plotstats}
\makeatother


% URLs
\usepackage[pdfpagelabels=false,pageanchor=false,bookmarks=false]{hyperref}

% Some abbreviations
\newcommand*{\Prob}{\mathbb{P}}
\newcommand*{\ECE}{\text{ECE}}
\newcommand*{\kernelmeasure}{\text{KCE}}
\newcommand*{\squaredkernelmeasure}{\text{SKCE}}

% Highlighting
\newcommand{\hl}[1]{\begingroup\bfseries\boldmath\color{uured}#1\endgroup}

% Source code highlighting
\tcbuselibrary{minted}
\newtcblisting{juliaconsnippet}[1][]{%
  enhanced jigsaw,
  listing engine=minted,
  minted language={lexer.py:Julia1ConsoleLexer -x},
  minted options={autogobble,breaklines,mathescape,fontsize=\footnotesize},
  minted style={colorful},
  listing only,
  breakable,
  sharp corners,
  colback=white,
  boxrule=1pt,
  leftrule=0pt,
  rightrule=0pt,
  left=0pt,
  top=0pt,
  bottom=0pt,
  right=0pt,
  #1
}

% Draw scratch counts
\usepackage{luamplib}
\newcommand{\scratchcount}[1]{%
\begin{mplibcode}
  beginfig(0);
  n:= #1;
  height := 3/5\mpdim{\normalbaselineskip};
  span := 1/3 * height ;
  drift := 1/10 * height ;
  pickup pencircle scaled (1/12 * height) ;
  def d = (uniformdeviate drift) enddef ;
  for i := 1 upto n :
    draw
      if (i mod 5)=0 : ((-d-4.5span,d)--(+d-0.5span,height-d))
      else : ((-d,+d)--(+d,height-d)) fi
      shifted (span*i,d-drift) ;
  endfor;
  endfig;
\end{mplibcode}}

% Download picture from the web, if necessary
\IfFileExists{./figures/penguins.png}{}{%
  \write18{curl -o ./figures/penguins.png --create-dirs https://raw.githubusercontent.com/allisonhorst/palmerpenguins/69530276d74b99df81cc385f4e95c644da69ebfa/man/figures/lter_penguins.png}
}

% Metadata
\title{Calibration tests beyond classification}
\author{David Widmann$^\star$ Fredrik Lindsten$^\ddagger$ Dave Zachariah$^\star$}
\date{}
\makeatletter
\pgfkeys{%
  /my poster/.cd,
  title/.initial=\@title,
  author/.initial=\@author,
  institute/.initial={},
  contact/.initial={},
  date/.initial=\@date,
}
\makeatother

\pgfkeys{%
  /my poster/.cd,
  institute={$^\star$Department of Information Technology, Uppsala University, Sweden\\$^\ddagger$Division of Statistics and Machine Learning, Linköping University, Sweden},
  contact={david.widmann@it.uu.se fredrik.lindsten@liu.se dave.zachariah@it.uu.se},
}

\pagestyle{empty}

\begin{document}
\NoHyper

\begin{tcbposter}

  % title
  \posterbox[blankest,interior engine=path,halign=flush center,valign=top,
  underlay =
  {%
    \node[below right,inner sep=0pt,outer sep=0pt] at (frame.north west) {\includegraphics[width=2cm]{../shared/figures/logos/UU.pdf}};%
    \node[above right,inner sep=0pt,outer sep=0pt] at (frame.south west) {\includegraphics[width=2cm]{../shared/figures/logos/LiU.pdf}};%
    \node[below left,inner sep=0pt,outer sep=0pt] at (frame.north east) {\includegraphics[width=2cm]{../shared/figures/logos/ICLR.pdf}};%
    \node[above left,inner sep=0pt,outer sep=0pt] at (frame.south east) {\qrcode[height=2cm]{https://devmotion.github.io/Calibration\_ICLR2021/}};%
  }]{name=title,column=1,span=6,below=top}{%
    \Huge\textbf{\pgfkeysvalueof{/my poster/title}}\\[1ex]
    \large\pgfkeysvalueof{/my poster/author}\\[1ex]
    \normalsize\pgfkeysvalueof{/my poster/institute}%
  }%

  % footline
  \posterbox[blankest,top=2pt,bottom=2pt,valign=center,fontupper=\ttfamily\small,interior engine=path,interior style={color=uumidgrey}%
  ]{name=footline,column=1,span=10,above=bottom}{%
    \pgfkeysvalueof{/my poster/date}\hfill\pgfkeysvalueof{/my poster/contact}%
  }%

  \posterbox[adjusted title={Motivation - what is a calibrated model?}]{name=calibration,column=3,span=4,below=title}{
    \begin{tcolorbox}[colback=blondstark, halign=flush center]
      A \hl{calibrated model} yields predictions consistent with empirically observed frequencies.
    \end{tcolorbox}

    \tcbsubtitle{Example: Classification}

    Consider a probabilistic predictive model of penguin species.

    \begin{adjustbox}{valign=t}
      \begin{tikzpicture}
        \node (model) at (0, 0)
        {\includegraphics[height=1cm]{../shared/figures/model.pdf}};
        \node[above=4mm of model, font=\bfseries] (label) {Model $P$};

        \node[left=1cm of model] (input) {\includegraphics[width=1.5cm]{../shared/figures/measure.pdf}};
        \node[anchor=base, font=\bfseries] at (input|-label.base) {Input $X$};
        \draw[line] (input) -- (model);

        \node[right=1cm of model] (prediction) {\includegraphics[width=3cm]{./figures/penguins.png}};
        \node [font=\bfseries, align=center] at (label-|prediction) {Prediction $P_X$\\(distribution of target $Y$)};
        \node [below=0mm of prediction, font=\tiny, anchor=west] {Artwork by \texttt{@allison\_horst}};
        \draw [line] (model) -- (prediction);
      \end{tikzpicture}
    \end{adjustbox}%
    \hfill%
    \begin{minipage}[t]{0.3\linewidth}
      The predictions $P_X$ of the model are \hl{probability distributions}, e.g.:

      \begin{center}
        \begin{tabular}{@{}ccc@{}}
          \textcolor{Dark2-B}{\texttt{Adélie}} & \textcolor{Dark2-C}{\texttt{Chinstrap}} & \textcolor{Dark2-A}{\texttt{Gentoo}} \\ \midrule
          80\% & 10\% & 10\% \\
        \end{tabular}
      \end{center}
    \end{minipage}

    If the model is calibrated we know that for all inputs with this
    prediction the measurements correspond to an Adélie penguin 80\% of the time
    and a Chinstrap and Gentoo penguin 10\% of the time each.

    \begin{center}
      \begin{tikzpicture}
        \node (prediction) at (0, 0)
        {\begin{tabular}{@{}ccc@{}}
           \textcolor{Dark2-B}{\texttt{Adélie}} & \textcolor{Dark2-C}{\texttt{Chinstrap}} & \textcolor{Dark2-A}{\texttt{Gentoo}} \\ \midrule
           80\% & 10\% & 10\% \\
         \end{tabular}};
        \node[above=1mm of prediction, font=\bfseries] (label) {Prediction $P_X$};

        \node[right=1cm of prediction] (empirical)
        {\begin{tabular}{@{}ccc@{}}
            \textcolor{Dark2-B}{\texttt{Adélie}} & \textcolor{Dark2-C}{\texttt{Chinstrap}} & \textcolor{Dark2-A}{\texttt{Gentoo}} \\ \midrule
            \scratchcount{8} \ldots & \scratchcount{2} \ldots & \scratchcount{1} \ldots \\
         \end{tabular}};
        \node[anchor=base, font=\bfseries] at (label.base-|empirical) {Empirical frequency $\Prob(Y | P_X)$};

        \node[font=\boldmath\Huge, color=uured, align=center] (A) at ($(prediction.center)!0.5!(empirical.center)$) {$=$};
        \node[above=0mm of A,font=\bfseries\Huge, color=uured, align=center] {?};
      \end{tikzpicture}
    \end{center}
  }

  \posterbox[adjusted title={Calibration of general models}]{name=error,column=1,span=3,between=calibration and footline}{
    The notion of calibration can be generalized to \hl{all probabilistic predictive models}.

    \begin{tcolorbox}[colback=blondstark]
      A probabilistic predictive model $P$ is calibrated if
      \begin{equation*}
        \Prob(Y \,|\, P_X) = P_X \qquad \text{almost surely}.
      \end{equation*}
    \end{tcolorbox}

    This definition captures also weaker notions such as confidence calibration and
    applies to models with targets $Y$ in general spaces such as (but not limited to)
    \begin{tcbraster}[blankest, raster columns=3,raster equal height=rows, halign=flush center]
      \begin{tcolorbox}
        $\mathbb{N}_0$\\[\baselineskip]
        \begin{tikzpicture}[scale=0.5]
          \begin{axis}[
            range frame,
            xlabel=$X$,
            ylabel=$Y$,
            width=1.8\linewidth,
            domain=-20:60,
            bar width=2,
            ]
            \pgfmathsetseed{1234}
            \addplot+ [ybar, no marks, samples=20] ({x + 0.1*rand}, {round(50*rnd)});
          \end{axis}
        \end{tikzpicture}
      \end{tcolorbox}
      \begin{tcolorbox}
        $\mathbb{R}^d$\\[\baselineskip]
        \begin{tikzpicture}[scale=0.5]
          \begin{axis}[
            range frame,
            xlabel=$X$,
            ylabel=$Y$,
            width=1.8\linewidth,
            domain=-20:60,
            ]
            \addplot+ [no marks, samples=2] {5 + 0.1*x};

            \pgfmathsetseed{1234}
            \addplot+ [only marks, mark=*, mark size=0.75, samples=70] ({x + 0.1*rand}, {5 + 0.1*x + 3*rand});
          \end{axis}
        \end{tikzpicture}
      \end{tcolorbox}
      \begin{tcolorbox}
        graphs\\[\baselineskip]
        \begin{tikzpicture}[scale=0.5]
          \pgfmathsetseed{1234}
          \graph [spring layout, nodes={draw, scale=0.5, circle, fill=Dark2-B, as=}, n=25, p=0.3] {
            subgraph I_n;
            subgraph G_np
          };
        \end{tikzpicture}
      \end{tcolorbox}
    \end{tcbraster}

    However, for general models the conditional distributions $\Prob(Y \,|\, P_X)$
    \begin{itemize}
    \item can be \hl{arbitrarily complex}---it is not a design choice!
    \item are \hl{difficult to estimate}---usually a single observation for each prediction!
    \end{itemize}

    \tcbsubtitle{Alternative formulation}

    We propose the following reformulation of the calibration property:

    \begin{tcolorbox}[colback=blondstark]
      A model $P$ is calibrated if $(P_X, Y) \stackrel{d}{=} (P_X, Z_X)$ where $Z_X \,|\, P_X \sim P_X$.
    \end{tcolorbox}

    \begin{itemize}
    \item No explicit conditional distributions $\Prob(Y \,|\, P_X)$
    \item Suggests discrepancy of $\Prob(P_X, Y)$ and $\Prob(P_X, Z_X)$
      as calibration measure
    \item Hypothesis testing of calibration is a special two-sample problem
    \end{itemize}
  }

  \posterbox[adjusted title=The paper in 30 seconds, fontupper=\large, colback=blondmellan]{name=summary,column=1,span=2,between=title and error}{
    \begin{itemize}
    \item We propose \hl{calibration errors} and their estimators and
      \hl{calibration tests} for \hl{all probabilistic predictive models}.
    \item For classification models, existing methods are reformulated
      more intuitively.
    \item We demonstrate calibration analysis for models with Gaussian
      predictions and ensembles of such models.
    \end{itemize}

    \tcbsubtitle{Take with you}
    \begin{itemize}
    \item First unified framework for calibration analysis of general predictive models
    \item Calibration tests are special two-sample problems
    \item Readily available software packages with estimators and hypothesis tests
    \end{itemize}
  }

  \posterbox[adjusted title={Calibration analysis}]{name=analysis,column=4,span=3,between=calibration and footline}{
    \tcbsubtitle{Calibration errors}

    To minimize the assumptions about the involved distributions, we use integral
    probability metrics to define a general class of calibration errors. The $\kernelmeasure$
    is an example of this class:

    \begin{tcolorbox}[colback=blondstark]
      The \hl{kernel calibration error} ($\kernelmeasure$) of model $P$ with respect to
      a real-valued kernel $k$ is defined as
      \begin{equation*}
        \kernelmeasure_k = \text{MMD}_k\big(\Prob(P_X, Y), \Prob(P_X, Z_X)\big).
      \end{equation*}
    \end{tcolorbox}

    \begin{itemize}
    \item Applies to \hl{all probabilistic predictive models}
    \item Coincides with previously proposed $\kernelmeasure$ for classification models
      but uses \hl{more intuitive real-valued kernels} instead of matrix-valued kernels
    \item Existing \hl{(un)biased and consistent estimators} of the MMD can be
      \hl{improved by marginalizing} out $Z_X$ and \hl{avoid challenging estimation} of $\Prob(Y|P_X)$
    \end{itemize}

    \tcbsubtitle{Calibration tests}

    \begin{minipage}[t]{0.4\linewidth}
      In general, calibration errors have no meaningful unit or scale.
      This renders it difficult to interpret an estimated non-zero error.

      We can use the calibration error estimates to perform a
      statistical test of the null hypothesis
    \end{minipage}%
    \hfill%
    \begin{adjustbox}{valign=t}
      \begin{tikzpicture}[
        scale=0.5,
        declare function={normal(\m,\s)=1/(2*\s*sqrt(pi))*exp(-(x-\m)^2/(2*\s^2));},
        declare function={binormal(\ma,\sa,\mb,\sb,\p)=(\p*normal(\ma,\sa)+(1-\p)*normal(\mb,\sb));}
        ]

        \begin{axis}[
          range frame,
          domain = -0.1:0.2,
          no marks,
          xlabel = calibration error estimate,
          ylabel = density,
          %grid=major,
          ymin = 0,
          tick label style={font=\tiny},
          label style={font=\small},
          width = \textwidth,
          height = 0.38\textwidth,
          name = axis,
          % legend cell align=left,
          legend columns=2,
          legend to name=pvalues,
          legend style=
          {
            fill=none,
            draw=none,
            inner sep={0pt},
            font=\small,
          }
          ]

          \draw[Dark2-A, thick] ({axis cs:0.07, 0}|-{rel axis cs:0,1}) -- ({axis cs:0.07,0}|-{rel axis cs:0,0}) node [at end, above, anchor=south east, sloped, font=\small] {observed};
          \draw[Dark2-B, thick] ({axis cs:0, 0}|-{rel axis cs:0,1}) -- ({axis cs:0,0}|-{rel axis cs:0,0}) node [at end, above, anchor=south east, sloped, font=\small] {calibrated};

          % mixture model of normal distributions
          \addplot+ [color=Dark2-B, dashed, thick, samples=31, smooth, name path=A] {binormal(-0.05,0.01,0.05,0.03,0.5)};
          \addlegendentry{distribution under $H_0$};

          % indicate p-value
          \path [name path=B] ({rel axis cs:0,0}|-{axis cs:0,0}) -- ({rel axis cs:1,0}|-{axis cs:0,0});
          \addplot+ [draw=Dark2-C, pattern color=Dark2-C, pattern={north east lines}] fill between [of=A and B, soft clip={domain=0.07:0.2}];
          \addlegendentry{p-value};
        \end{axis}

        \node[anchor=south] at ($(axis.outer north)$) {\pgfplotslegendfromname{pvalues}};
      \end{tikzpicture}
    \end{adjustbox}
    \begin{equation*}
      H_0 \coloneqq \text{\enquote{the model is calibrated}}.
    \end{equation*}

    \begin{itemize}
    \item Applies to \hl{all probabilistic predictive models}
    \item Existing \hl{two-sample tests} based on the MMD can be
      \hl{improved by marginalizing} out $Z_X$
    \end{itemize}
  }

  \posterbox[adjusted title={Experiment: Friedman 1 regression problem}]{name=experiment,column=7,span=4,below=top}{
    \tcbsubtitle{Regression problem}

    \begin{itemize}
    \item Real-valued targets $Y$ depend on inputs $X \sim U\big([0,1]^{10}\big)$ via
      \begin{equation*}
        Y = 10 \sin{(\pi X_1 X_2)} + 20{(X_3 - 0.5)}^2 + 10 X_4 + 5 X_5 + \epsilon
      \end{equation*}
      with standard normally distributed noise $\epsilon \sim \mathcal{N}(0, 1)$
    \item Training and test dataset of 100 and 50 i.i.d.\ samples, respectively
    \end{itemize}

    \tcbsubtitle{Modeling setup}

    \begin{itemize}
    \item Predictive model $P_X \coloneqq \mathcal{N}(f_\theta(X), \sigma^2)$, where $f_\theta$ is modeled by a fully connected neural network with parameters $\theta$ (2 hidden layers, ReLU activation functions)
    \item Parameters $\theta$ and $\sigma$ optimized with a maximum likelihood approach
      for 10 random initializations $\theta_0$ and $\sigma_0$
    \end{itemize}

    \tcbsubtitle{Evaluation}

    In addition to the individual models, we also evaluate their ensemble.

    We use consistent estimators of the squared $\kernelmeasure$ ($\squaredkernelmeasure$)
    to estimate calibration of the models. The kernel is chosen as
    \begin{equation*}
      k\big((p, y), (p', y')) \coloneqq \exp{\big(-W_2(p, p')\big)} \exp{\big(- (y - y')^2/2\big)}
    \end{equation*}
    for targets $y, y' \in \mathbb{R}$ and predicted normal distributions $p, p'$, where
    $W_2$ is the 2-Wasserstein distance.

    \begin{center}
      \pgfplotsset{width=0.3\textwidth, height=0.25\textwidth}
      \begin{tikzpicture}
        \begin{groupplot}[
          range frame,
          group style={
            group size={3 by 1},
            xticklabels at={edge bottom},
            ylabels at={edge left},
            horizontal sep=5em,
          },
          xlabel=iteration,
          ylabel=estimate,
          no markers,
          width=0.3\textwidth,
          height=0.25\textwidth,
        ]

          % Plot NLL with legend
          \nextgroupplot[
            title=NLL,
            y coord trafo/.code=\pgfmathparse{arcsinh(#1)},
            yticklabel={$\sinh(\pgfmathprintnumber{\tick})$},
            legend columns=3,
            legend to name=leg,
          ]
          \plotstats{NLL}
          \legend{range (train),mean (train),ensemble (train),range (test),mean (test),ensemble (test)}

          % Plot SKCE
          \nextgroupplot[title=SKCE, ymode=log]
          \plotstats[SKCE (biased)]{SKCE}

          % Plot pvalues
          \nextgroupplot[title=p-value]
          \plotstats[p-value]{pvalue}
        \end{groupplot}
        % Add legend
        \node [anchor=north] at ($(group c1r1.west |- group c1r1.outer south)!0.5!(group c3r1.east |- group c3r1.outer south)$){\pgfplotslegendfromname{leg}};
      \end{tikzpicture}

      \begin{tcolorbox}[colback=blondstark, halign=flush center]
        NLL measures both calibration and resolution
      \end{tcolorbox}
    \end{center}
  }

  \begin{posterboxenv}[adjusted title={Software packages}]{name=software,column=7,span=4,between=experiment and footline}{}
    We provide software packages for calibration analysis.

    \tcbsubtitle{Julia}

    \begin{tcbraster}[blankest, raster columns=2, raster equal height=rows, raster force size=false, raster column skip=5mm]
      \begin{tcolorbox}[add to width=-3mm]
        \begin{itemize}
        \item \texttt{CalibrationErrors.jl}:\\
          Estimators of calibration errors such as $\ECE$ and $\squaredkernelmeasure$
        \item \texttt{CalibrationTests.jl}:\\
          Statistical hypothesis tests
        \item \texttt{CalibrationErrorsDistributions.jl}:\\
          Definitions for more general predictive models
          such as normal distributions and mixture models
        \end{itemize}
        \begin{minipage}[t]{\linewidth-3\baselineskip}
          Documentation available at:\\
          \url{https://devmotion.github.io/CalibrationErrors.jl/dev}
        \end{minipage}%
        \hfill%
        \adjustbox{valign=t}{\qrcode[height=3\baselineskip]{https://devmotion.github.io/CalibrationErrors.jl/dev}}
      \end{tcolorbox}
      \begin{juliaconsnippet}[add to width=3mm]
julia> AsymptoticSKCETest(kernel, predictions, targets)
Asymptotic SKCE test
--------------------
Population details:
    parameter of interest:   SKCE
    value under h_0:         0.0
    point estimate:          0.0263758

Test summary:
    outcome with 95% confidence: reject h_0
    one-sided p-value:           <1e-99

Details:
[...]
      \end{juliaconsnippet}
    \end{tcbraster}

    \tcbsubtitle{Python \& R}

    \begin{itemize}
    \item \texttt{pycalibration}: Python interface of the Julia packages (based on \texttt{pyjulia})
    \item Interface for R will be available soon (will be based on \texttt{JuliaCall})
    \end{itemize}
  \end{posterboxenv}
\end{tcbposter}
\end{document}
